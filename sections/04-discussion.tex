Based on the results outlined in the previous section, we have a reasonable insight into the thermochemical properties of the reactions without touching a single beaker. This approach ca be extremely useful as an initial guess at some reaction properties.

However, all the values had some level of error from the experimental values. Some of them actually had very large amount of deviation and some had lower. This is to  be expected in a computational approach as all algorithms rely on approximations and those approximations introduce a level of error. However, the Density Functional Theory-based B3LYP method consistently outperformed the other two methods. In this case, it had the lowest error in all three  reactions.

The accuracy of DFT can be attributed to the fact that it is the method that tries to include the most number of covariates. In this case, it triumphs over Hartree-Fock based methods by including the terms for electron correlation. This, when combined with a polarised and diffused triple-zeta basis set, allows it to outperform other calculations.

One thing that we can see clearly from this dataset is the stabilization effect of conjugated systems. Conjugated systems are systems with overlapping p-orbitals. This overlap allows the $\pi$ electrons to delocalize and increase the stability \cite{smith_marchs_2006}. In this case, we can see that both in case of benzene and the isopropylium cation, the change in enthalpy was significantly greater than the ones seen in the transfer of hydroxyl functional group. Thus, it can be reasonably concluded that conjugation confers an increased level of stability.

The data we obtained gives us a lot of information to calculate things like enthalpy and heat capacity. However, there is still a piece missing should you need to calculate the Gibbs free energy. That is the Total Entropy of the system. Once calculated, we can simply use Equation \ref{eq:gibbs} to calculate the Gibbs free energy.

\begin{equation}\label{eq:gibbs}
    G = H - TS_{tot}
\end{equation}