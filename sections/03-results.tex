\subsection{Equation 0}

Equation zero is given below:

\begin{center}
    \ce{C6H6 + 6CH4 -> 3C2H4 + 3C2H6}
\end{center}

Using the process outlined in the Methods section, we were able to calculate the values of Enthalpy change for the reaction. The source values used for each theory level and molecule are given in table \ref{tab:reaction-0-raw}. The final values of enthalpy, along with experimentally deternmined values and the associated error are give in Table \ref{tab:reaction-0-enthalpy}.

\import{tables/}{equation-0-raw}

\begin{table}[htbp]
\centering
\caption{The calculated enthalpy for Equation 0}
\label{tab:reaction-0-enthalpy}
\begin{tabular}{|c|c|c|c|c|}
\hline
      & Calculated & Uncorrected & Experimental & Percentage \\ \hline
B3LYP & 68.0514377 & 65.0092741  & 65.04        & 4.63013184 \\ \hline
HF    & 63.2227443 & 59.1037752  & 65.04        & -2.7940585 \\ \hline
MP2   & 64.0684367 & 59.7273294  & 65.04        & -1.4937935 \\ \hline
\end{tabular}
\end{table}

\subsection{Equation 1}

Equation one is given below:

\begin{center}
    \ce{CH4 + CH3CH2OH -> CH3CH3 + CH3OH}
\end{center}

Using the process outlined in the Methods section, we were able to calculate the values of Enthalpy change for the reaction. The source values used for each theory level and molecule are given in table \ref{tab:reaction-1-raw}. The final values of enthalpy, along with experimentally deternmined values and the associated error are give in Table \ref{tab:reaction-1-enthalpy}.

\import{tables/}{equation-1-raw}

\begin{table}[htbp]
\centering
\caption{The calculated enthalpy for Equation 1}
\label{tab:reaction-1-enthalpy}
\begin{tabular}{|c|c|c|c|c|}
\hline
      & Calculated & Uncorrected & Experimental & Percentage \\ \hline
B3LYP & 3.4520078  & 2.90984002  & 4.85         & -28.824581 \\ \hline
HF    & 3.26411327 & 2.62719163  & 4.85         & -32.698696 \\ \hline
MP2   & 3.33272468 & 2.64748486  & 4.85         & -31.284027 \\ \hline
\end{tabular}
\end{table}

\subsection{Equation 2}

Equation two is given below:

\begin{center}
    \ce{C3H7+ + CH4 -> C3H8 + CH3+}
\end{center}

Using the process outlined in the Methods section, we were able to calculate the values of Enthalpy change for the reaction. The source values used for each theory level and molecule are given in table \ref{tab:reaction-2-raw}. The final values of enthalpy, along with experimentally deternmined values and the associated error are give in Table \ref{tab:reaction-2-enthalpy}.

\import{tables/}{equation-2-raw}

\begin{table}[htbp]
\centering
\caption{The calculated enthalpy for Equation 2}
\label{tab:reaction-2-enthalpy}
\begin{tabular}{|c|c|c|c|c|}
\hline
      & Calculated & Uncorrected & Experimental & Percentage \\ \hline
B3LYP & 65.660794  & 64.787929   & 64.9        & 1.172255829 \\ \hline
HF    & 51.0817257 & 50.4247238  & 64.9        & -21.29163988 \\ \hline
MP2   & 51.5445172 & 50.3924106  & 64.9        & -20.57855598 \\ \hline
\end{tabular}
\end{table}

\subsection{Time Complexity}

Table \ref{tab:time} summarizes the time it took to run each molecule through the \emph{OPT + FREQ} process for each level of theory.

\begin{table}[htbp]
\centering
\caption{Time taken to complete the calculations for each molecule}
\label{tab:time}
\begin{tabular}{|l|l|l|l|}
\hline
Molecule     & B3LYP     & HF       & MP2       \\ \hline
Benzene      & 10m 40.8s & 6m 20.2s & 34m 52.1s \\
Ethane       & 0m 26.3s  & 0m 12.5s & 0m 48.4s  \\
Ethanol      & 1m 58.4s  & 0m 46s   & 3m 26.8s  \\
Ethylene     & 0m 19.2s  & 0m 11s   & 0m 27.9s  \\
Isopropylium & 8m 42.4s  & 2m 18.6s & 9m 30.1s  \\
Methane      & 0m 10.4s  & 0m 6.7s  & 0m 14.1s  \\
Methanol     & 0m 49.2s  & 0m 16s   & 0m 45.3s  \\
Methylium    & 0m 8.2s   & 0m 6.8s  & 0m 11.7s  \\
Propane      & 1m 54s    & 0m 58s   & 4m 21.5s  \\ \hline
\end{tabular}
\end{table}