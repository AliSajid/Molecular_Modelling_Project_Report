\subsection{Reactions}

For this paper, we selected three reactions, tentatively titled as \emph{Equation 0}, \emph{Equation 1} and \emph{Equation 2}. All of these equations relate to organic chemistry.

Equation 0: \ce{C6H6 + 6CH4 -> 3C2H4 + 3C2H6}

Equation 1: \ce{CH4 + CH3CH2OH -> CH3CH3 + CH3OH}

Equation 2: \ce{C3H7+ + CH4 -> C3H8 + CH3+}

\subsection{Method}

All analysis was performed using Gaussian 09 through the WebMO interface \cite{polik_webmo_nodate}. There were 8 unique structures. Each structure was created in the molecule builder of WebMO. It was then cleaned up by using the "Comprehensive - Mechanics" option in the window. The molecule was run with the OPT + FREQ to optimize the structure and determine vibrational frequencies.

All calculations were performed using the 6-311++G** basis set using three representative levels of theory.

\begin{itemize}
    \item Hartree-Fock - HF
    \item Post Hartree-Fock - MP2
    \item Density Functional Theory - B3LYP
\end{itemize}

\subsection{Calculations}

The calculations were done using Equation \ref{eq:enthalpy_calc}. This was then cross-checked by using the NIST WebBook \cite{linstrom_nist_1997} for the associated thermochemistry data and the reported enthalpy values from gaussian output. The equations are referenced from the guassian whitepaper on thermochemistry \cite{ochterski_thermochemistry_2012}.

\begin{equation} \label{eq:enthalpy_calc}
    \Delta_r H^{\circ} (298K) = \sum_{products} \Delta_f H^{\circ} (298k) - \sum_{products} \Delta_f H^{\circ} (298k)
\end{equation}

For each substance involved, the following parameters were extracted from the Gaussian output and converted to KCal/mol for standardization.

\begin{itemize}
    \item Ground State Energy ($E$)
    \item Energy Correction ($E_{corr}$)
    \item Zero-point energy ($ZPE$)
    \item The Vibration Frequencies ($\nu_i$)
\end{itemize}

The vibrational energy correction for the 298K Temperature was obtained by using Equation \ref{eq:vibration_calc}.

\begin{equation} \label{eq:vibration_calc}
    E_{vib}(T) = RT \sum_i \frac{x_i}{e^{x_i} - 1}
\end{equation}

where

\begin{equation}
    x_i = \frac{h \nu_i}{kT}
\end{equation}

The final form of the energy of each substance is given in Equation \ref{eq:energy_calc} and the final equation used for the calculation of Enthalpy in Equation \ref{eq:final_calc}.

\begin{equation} \label{eq:energy_calc}
    E_{Total} = E + E_{corr} + ZPE + E_{vib}
\end{equation}

\begin{equation} \label{eq:final_calc}
    \Delta H^{\circ}_{298} = \Delta E_{Total} + 4RT
\end{equation}

\subsection{Reference Data}

For reference, the heats of formation of each species involved were obtained. The data is summarised in Table \ref{tab:reference_data}. The data was sourced from \cite{linstrom_nist_1997, pittam_measurements_1972, chang_vacuum_2017, roux_critically_2008, manion_evaluated_2002}. 

\begin{table}[htbp]
\centering
\caption{The heats of formation of each species in KCal/mol}
\label{tab:reference_data}
\begin{tabular}{|c|c|c|}
\hline
Molecule     & Formula & Hf      \\ \hline
Benzene      & \ce{C6H6}    & 19.8    \\ \hline
Ethane       & \ce{C2H6}    & -20.04  \\ \hline
Ethanol      & \ce{C2H5OH}  & -56     \\ \hline
Ethylene     & \ce{C2H4}    & 12.54   \\ \hline
Isopropylium & \ce{C3H7+}   & 192.591 \\ \hline
Methane      & \ce{CH4}     & -17.89  \\ \hline
Methanol     & \ce{CH3OH}   & -49     \\ \hline
Methylium    & \ce{CH3+}    & 34.821  \\ \hline
Propane      & \ce{C3H8}    & -25.02  \\ \hline
\end{tabular}
\end{table}